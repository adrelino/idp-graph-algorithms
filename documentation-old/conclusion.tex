%% conclusion.tex
%%

%% ==================
\chapter{Further Reading}
\label{ch:reader}
%% ==================

In this chapter we collect some suggested literature for \LaTeX{} matters, which may be of use for beginners and for more advanced users and may provide some useful tips.


\begin{description}
\item[lshort:] \enquote{The Not So Short Introduction to \LaTeX}
  (see \cite{l2short}) is an up-to-date introduction which can be worked through in a moderate amount of time (the authors give an estimated time of 157 minutes for version 5.01, the most recent at the time of writing.) An up-to-date version can be found at \url{http://tobi.oetiker.ch/lshort/lshort.pdf}.

\item[\LaTeX{} and Friends:] The book \cite{vanDongen2012} is a recommended and up-to-date introduction to \LaTeX{} which addresses many current packages. Worth a look for both beginners and advanced users.


\item[l2tabu:] There are many tips for older packages and \LaTeX{} commands in \cite{l2tabu}, which are particularly recommended for advanced \LaTeX{} users. Here you can learn why certain commands are best avoided and what the alternatives are. Note: The \texttt{tumthesis.cls} class automatically calls upon the \texttt{nag} package, which immediately rings alarm bells with many of the mistakes listed in l2tabu.


\end{description}

%%% Local Variables: 
%%% mode: latex
%%% TeX-master: "thesis"
%%% End: 