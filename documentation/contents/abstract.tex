%==================================================
% abstract.tex
% Beispieldatei für tumthesis.cls und thesis.tex
% Michael Ritter, 2012
% Lizenz: 
% This work may be distributed and/or modified under the
% conditions of the LaTeX Project Public License, either version 1.3
% of this license or (at your option) any later version.
% The latest version of this license is in
% http://www.latex-project.org/lppl.txt
% and version 1.3 or later is part of all distributions of LaTeX
% version 2005/12/01 or later.
%==================================================
\cleardoublepage

\selectlanguage{english}
\section*{Abstract}
This interdisciplinary project deals with the vivid visualization of advanced graph algorithms. In particular, algorithms to solve two distinctive problems in discrete math are considered. Namely the \maxflow{} as well as the \spprc{}. For efficiency reasons, one often employs advanced graph algorithms to solve above problems. The \maxflow{} is solved using the efficient \pushRelabel{}. To solve the \spprc{}, we employ a generic \labelSetting{} which follows the dynamic programming principle.


Both algorithms carry a lot of state variables and are thus not easy to understand intuitively. An additional visualization layer with an intuitive representation of all state variables and state transitions during algorithm execution was developed. It displays the height function of each node in case of the \pushRelabel{} or the pareto frontier of all labels resident in a certain node in case of the \labelSetting{}. To achieve the goal of a high interactivity, we replaced the previous Canvas based graph visualization code with a new implementation based on SVG, using D3.js, a JavaScript library for producing dynamic, interactive data visualizations in web browsers.


%This IDP aims at extending the previous HTML5 framework that was developed at the chair to support advanced graph algorithms:
%\begin{itemize}
%	\item the Tarjan-Goldberg push-relabel algorithm to solve the maximum flow problem
%	\item a Label Setting algorithm to solve the shortest path problem with resource constraints (SPPRC)
%\end{itemize}
%In particular, the goal was to provide an intuitive visual representation of all state variables and state transitions during the algorithm execution. Since both algorithms carry a lot of state information, an additional visualization layer, linked to the original graph layer was developed. This second layer displays: 
%\begin{itemize}
%	\item the height function of each node in case of the Tarjan-Goldberg algorithm
%	\item the pareto frontier of all labels resident in a certain node in case of the Label Setting algorithm
%\end{itemize}
%
%To achieve the goal of a highly interactive and easily extensible user experience, we replaced the old canvas based graph visualization code, which was really hard to extend, with a new implementation using the Stanford development D3.js (or just D3 for Data-Driven Documents), a JavaScript library for producing dynamic, interactive data visualizations in web browsers. It makes use of the widely implemented SVG, HTML5, and CSS standards. 
%
%During the development process, we also migrated the existing graph editor to the new technologies, it now easily supportsan arbitrary number of resources defined on edges and nodes download and upload functionality cropped SVG export of graphs
%
%This talk is not only suitable to mathematicians found of graph problems, but to anyone who wants to leverage today's web standards to express his interactive visualization needs.
%

\selectlanguage{ngerman}
\section*{Zusammenfassung}
Das vorliegende interdisziplinäre Projekt beschäftigt sich mit der anschaulichen Darstellung von fortgeschrittenen Graphalgorithmen. Betrachtet werden zwei Verfahren zur Lösung von Problemstellungen der diskreten Mathematik. Die zu visualisierenden Problemstellungen sind hierbei das \maxflowDE{} sowie das \spprcDE{}. Als Lösungsverfahren für die aufgeführten Problemstellungen werden aus Effizienzgründen häufig fortgeschrittene Graphalgorithmen herangezogen. Für die Lösung des \maxflowDEGen{} findet der bekannte \pushRelabelDE{} in der Praxis häufig Anwendung. Zur Lösung des \spprcDEGen{} wird mit einem \labelSettingDE{} ein bekanntes Verfahren der dynamischen Programmierung vorgestellt.


Beide Algorithmen führen eine Menge an Statusvariablen mit sich und sind deshalb nicht leicht intuitiv zu verstehen. Es wurde eine zusätzliche Visualisierungsebene mit intuitiver Repräsentation aller Statusvariablen und -übergänge entwickelt. Diese veranschaulicht die Höhenfuntion jedes Knotens im Falle des \pushRelabelDE{} oder dessen Pareto-Front im Falle des \labelSettingDE{}. Um hohe Interaktivität zu erreichen ersetzten wir den bisherigen auf Canvas basierenden Graphen-Visualisierungscode mit einer neuen Implementierung basierend auf SVG mit D3.js, eine JavaScript Bibliothek zur Erzeugung dynamischer, interaktive Datenvisualisierungen in Web Browsern.
%Das interdisziplinäre Projekt hat das Ziel, die erwähnten Problemstellungen mit einfachen Worten zu erklären und durch Beispiele zu motivieren sowie die verwendeten Lösungsverfahren graphisch zu veranschaulichen. Die Darstellung wird hierbei in Form einer separaten Web-Applikation für jede Problemstellung erfolgen, welche aus Gründen der einheitlichen Darstellung und der der leichteren Wartbarkeit auf ein gemeinsames, bereits existierendes Framework aufbauen wird.

\selectlanguage{english}

%%% Local Variables: 
%%% mode: latex
%%% TeX-master: "thesis"
%%% End: 