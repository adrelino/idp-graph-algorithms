%==================================================
% abstract.tex
% Beispieldatei für tumthesis.cls und thesis.tex
% Michael Ritter, 2012
% Lizenz: 
% This work may be distributed and/or modified under the
% conditions of the LaTeX Project Public License, either version 1.3
% of this license or (at your option) any later version.
% The latest version of this license is in
% http://www.latex-project.org/lppl.txt
% and version 1.3 or later is part of all distributions of LaTeX
% version 2005/12/01 or later.
%==================================================
\cleardoublepage

\selectlanguage{english}
\section*{Abstract}
This interdisciplinary project deals with the vivid visualization of advanced graph algorithms. In particular, algorithms to solve two distinctive problems in discrete math are considered. Namely the \texttt{max-flow problem} as well as the \texttt{shortest path problem with resource constraints}.
For efficiency reasons, one often employs advanced graph algorithms to solve above problems. The \texttt{max-flow problem} is solved using the efficient push-relabel algorithm of Goldberg-Tarjan. To solve the \texttt{shortest path problem with resource constraints}, we employ a generic label-setting algorithm which follows the dynamic programming principle.


Both algorithms carry a lot of state variables and are thus not easy to understand intuitively. An additional visualization layer with an intuitive representation of all state variables and state transitions during algorithm execution was developed. It displays the height function of each node in case of the Tarjan-Goldberg algorithm or the pareto frontier of all labels resident in a certain node in case of the label-setting algorithm. To achieve the goal of a high interactivity, we replaced the previous Canvas based graph visualization code with a new implementation based on SVG, using D3.js, a JavaScript library for producing dynamic, interactive data visualizations in web browsers.


%This IDP aims at extending the previous HTML5 framework that was developed at the chair to support advanced graph algorithms:
%\begin{itemize}
%	\item the Tarjan-Goldberg push-relabel algorithm to solve the maximum flow problem
%	\item a Label Setting algorithm to solve the shortest path problem with resource constraints (SPPRC)
%\end{itemize}
%In particular, the goal was to provide an intuitive visual representation of all state variables and state transitions during the algorithm execution. Since both algorithms carry a lot of state information, an additional visualization layer, linked to the original graph layer was developed. This second layer displays: 
%\begin{itemize}
%	\item the height function of each node in case of the Tarjan-Goldberg algorithm
%	\item the pareto frontier of all labels resident in a certain node in case of the Label Setting algorithm
%\end{itemize}
%
%To achieve the goal of a highly interactive and easily extensible user experience, we replaced the old canvas based graph visualization code, which was really hard to extend, with a new implementation using the Stanford development D3.js (or just D3 for Data-Driven Documents), a JavaScript library for producing dynamic, interactive data visualizations in web browsers. It makes use of the widely implemented SVG, HTML5, and CSS standards. 
%
%During the development process, we also migrated the existing graph editor to the new technologies, it now easily supportsan arbitrary number of resources defined on edges and nodes download and upload functionality cropped SVG export of graphs
%
%This talk is not only suitable to mathematicians found of graph problems, but to anyone who wants to leverage today's web standards to express his interactive visualization needs.
%

\selectlanguage{ngerman}
\section*{Zusammenfassung}
Das vorliegende interdisziplin\"are Projekt besch\"aftigt sich mit der anschaulichen Darstellung von fortgeschrittenen Graphalgorithmen. Betrachtet werden zwei Verfahren zur L\"osung von Problemstellungen der diskreten Mathematik. Die zu visualisierenden Problemstellungen sind hierbei das \texttt{Max-Flow Problem} sowie das \texttt{K\"urzeste-Wege Problem mit Ressourcenbeschr\"ankungen}. Als L\"osungsverfahren f\"ur die aufgefu\"uhrten Problemstellungen werden aus Effizienzgr\"unden h\"aufig fortgeschrittene Graphalgorithmen herangezogen. F\"ur die L\"osung des Max-Flow Problems findet der bekannte Push-Relabel Algorithmus von Goldberg-Tarjan in der Praxis h\"aufig Anwendung. Zur L\"osung des K\"urzeste-Wege Problems mit Ressourcenbeschr\"ankungen wird mit einem Label-Setting Algorithmus ein bekanntes Verfahren der dynamischen Programmierung vorgestellt.


Beide Algorithmen f\"uhren eine Menge an Statusvariablen mit sich und sind deshalb nicht leicht intuitiv zu verstehen. Es wurde eine zus\"atzliche Visualisierungsebene mit intuitiver Repr\"asentation aller Statusvariablen und -\"uberg\"ange entwickelt. Diese veranschaulicht die H\"ohenfuntion jedes Knotens im Falle des Tarjan-Goldberg Algorithmus oder die Pareto-Front im Falle des Label-Setting Algorithmus. Um hohe Interaktivit\"at zu erreichen ersetzten wir den bisherigen auf Canvas basierenden Graphen-Visualisierungscode mit einer neuen Implementierung basierend auf SVG mit D3.js, eine JavaScript Bibliothek zur Erzeugung dynamischer, interaktive Datenvisualisierungen in Web Browsern.
%Das interdisziplinäre Projekt hat das Ziel, die erwähnten Problemstellungen mit einfachen Worten zu erklären und durch Beispiele zu motivieren sowie die verwendeten Lösungsverfahren graphisch zu veranschaulichen. Die Darstellung wird hierbei in Form einer separaten Web-Applikation für jede Problemstellung erfolgen, welche aus Gründen der einheitlichen Darstellung und der der leichteren Wartbarkeit auf ein gemeinsames, bereits existierendes Framework aufbauen wird.

\selectlanguage{english}

%%% Local Variables: 
%%% mode: latex
%%% TeX-master: "thesis"
%%% End: 